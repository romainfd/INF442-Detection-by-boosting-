\documentclass[a4paper,11pt]{article}
\usepackage[utf8]{inputenc}
\usepackage[french]{babel} 
\usepackage[T1]{fontenc} 
\usepackage{textcomp}
\usepackage{amsmath,amssymb}
\usepackage{mathrsfs}
\usepackage{stmaryrd}
\usepackage{graphicx}
\usepackage[titlepage,fancysections]{polytechnique}
\graphicspath{{Image/}}

\title{Détection par "boosting"}
\author{Romain FOUILLAND et Antoine GROSNIT}
\subtitle{PI - INF442}
\date{Mai 2018}

\begin{document}
\maketitle
\section{Informations pratiques}
Pour faire fonctionner notre programme, il suffit suivre les instructions disponibles dans le fichier readme.md ou de se rendre sur la page GitHub du projet : https://github.com/romainfd/INF442-Detection-by-boosting-/.

\section{Caractéristiques employées}
\subsection{Calcul de l'image intégrale}

Q1. Afin d'obtenir l'image intégrale à partir d'une matrice correspondant à une image, nous avons procédé, comme le suggère l’énoncé, par récurrence, en utilisant un principe de programmation dynamique.

\subsection{Création des classifieurs faibles}

Q2. Nous n’avons considéré ici que des caractéristiques « carré » (ayant le même nombre de pixels en longueur et en largeur). On obtient alors, pour une zone de $92 \times 112$, un côté minimal de $8$ pixels et un incrément de position et de taille de $4$ pixels, $4466$ caractéristiques (et de manière générale il y en a $O(n^3)$ où $n$ est la longueur de l'image) :

\begin{align*}
	\sum_{i=2}^{22} (28-i)(23-i) &= \sum_{i=1}^{21} (5+i)i \\
	&= 5\sum_{i=1}^{21} i + \sum_{i=1}^{21} i^2\\
	&= 5\times\frac{21 \times 22}{2} + \frac{21 \times 22 \times 43}{6} \\
	&= 4466
\end{align*}
Cette valeur a pu être confirmée par la fonction $nbCaractsTot$ du programme.

Le calcul des caractéristiques se fait de façon parallèle avec MPI :\\

\begin{itemize}
	\item Chaque processus calcule entièrement l’image intégrale pour pouvoir calculer des caractéristiques en temps constant. \\
	\item Chaque processus calcule, en fonction de son rang, certaines caractéristiques de l’image, de sorte que toutes les caractéristiques de l’image soient calculées par un unique processus. \\
	\item Le processus racine reçoit ($MPI\_Recv$) de la part de tous les processus dans l’ordre de leur rang, un tableau des caractéristiques qu’ils ont calculées, et les agrège dans cet ordre dans un vecteur d’entier.\\
	\item Ledit vecteur est retourné (on notera que quel que soit le processus racine, les caractéristiques retournées sont bien toujours dans le même ordre).\\
\end{itemize}

\section{Création d'un classifieur}
\subsection{Création des classifieurs faibles}
Q3. Dans le but d’entraîner nos classifieurs faibles de manière parallèle, nous avons procédé de la façon suivante avec MPI (avec p processus) : \\

\begin{itemize}
	\item Nous récoltons tout d’abord les titres des images présentes dans la base d’apprentissage, et nous les stockons dans un vecteur de $string$ pour pouvoir tirer consécutivement $K$ images au sort. \\
	\item  $w_1$ et $w_2$, vecteurs contenant les $w_{i1}$ et $w_{i2}$, sont initialisés.\\
	\item  Pour $k$ variant de 1 à $K$, nous tirons une image au sort (la même pour chaque processus grâce à l’initialisation commune de la $seed$).\\
	\item  Au bout de chaque série de $p$ passages dans la boucle, les $p$ processus ont appliqué une et une seule fois l’algorithme décrit dans la question précédente en tant que processus racine. Chaque processus dispose donc d’un vecteur complet de caractéristiques correspondant à une image distincte qui lui permet de calculer (suivant la logique décrite dans l’énoncé) la variation incrémentale de $w_1$ et $w_2$. On met alors à jour les valeurs de $w_1$ et $w_2$ pour tous les processus à l’aide d’un $MPI\_Allreduce$. Il faut noter que, contrairement à l'implémentation séquentielle pour laquelle les valeurs de $w_1$ et $w_2$ auraient été mises à jour à chaque boucle, ici nous ne procédons à cette mise à jour que tous les $p$ passages en agrégeant en une fois les $p$ modifications calculées en parallèles. \\
	\item  Les valeurs stockées dans $w_1$ et $w_2$, obtenues à l’issue des $K$ tirages au sort, déterminent nos classifieurs $h_i$.\\
		
\end{itemize}

Convergence des paramètres en fonction de $K$ et $\epsilon$ :\\
Nous avons considéré l'évolution des quantités $  \frac{\Vert(w_1^{k+1} - w_1^{k})\Vert}{\Vert w_1^{k}\Vert}$ et $  \frac{\Vert(w_2^{k+1} - w_2^{k})\Vert}{\Vert w_2^{k}\Vert}$ où $\Vert . \Vert$ désigne la norme euclidienne et $w_i^k$ désigne le vecteur $w_i$ à la boucle $k$. On mesure ainsi l'évolution relative des paramètres au cours de leur $K$ $mod$ $p$ mises à jour. 

\subsection{Choix de $K$}
En traçant l'évolution de $  \frac{\Vert(w_2^{k+1} - w_2^{k})\Vert}{\Vert w_2^{k}\Vert}$ pour $K = 200$, $2000$ et $10000$, on constate que cette quantité converge pour différentes valeurs de $\epsilon$ et que le comportement de $  \frac{\Vert(w_1^{k+1} - w_1^{k})\Vert}{\Vert w_1^{k}\Vert}$ est non convergent pour ces trois valeurs de $K$. Il semble donc judicieux de prendre $K$ de l'ordre de $2000$ pour limiter le nombre de boucles tout en ayant une convergence suffisante de $w_2$. \\

\includegraphics[width=15cm]{"Fig1 Cv K=200"}\\
\includegraphics[width=15cm]{"Fig2 Cv K=2000"}\\
\includegraphics[width=15cm]{"Fig3 Cv K=10000"}\\
	
\subsection{Choix de $\epsilon$}
En raisonnant sur les ordres de grandeurs des paramètres (et notamment de $X_{ki} \sim 10000$) nous avons distingué deux régimes distincts :\\
	
	\begin{itemize}
		\item Pour $\epsilon \gg \frac{1}{10000}$ : $w_{i2}$ est de l'ordre de $\epsilon$ et $w_{i1}$ est de l'ordre de $\epsilon \times X_i \sim 10000 \times \epsilon$. Ainsi, $w_{2i}$ ne joue globalement pas de rôle dans la classification qui est déterminée uniquement par le signe de $w_{i1} \gg w_{i2}$. De ce fait, la valeur particulière de $\epsilon$ dans ce régime a très peu d'impact sur le résultat. Etant donnée la variation des valeurs des $X_{ki}$ et de la catégorie de l'image tirée au sort à chaque boucle, la quantité $  \frac{\Vert(w_1^{k+1} - w_1^{k})\Vert}{\Vert w_1^{k}\Vert}$ reste de l'ordre de l'unité quel que soit $K$, et on n'observe pas de convergence.\\
		
		\item Pour $\epsilon \ll \frac{1}{10000}$ : $w_{i2}$ reste de l'ordre de $\epsilon$ et $w_{i1}$ est de l'ordre de l'unité (son initialisation). Là encore, $w_{2i}$ ne joue globalement pas de rôle dans la classification qui est déterminée uniquement par le signe de $w_{i1} \gg w_{i2}$ (car $1 \gg \epsilon$). De ce fait, la valeur particulière de $\epsilon$ dans ce régime a très peu d'impact sur le résultat (le classifieur est déterminé par la valeur de $w_{i1}$ qui varie très faiblement autour de sa valeur initiale, indépendante de $\epsilon$ - ce qui tend d'ailleurs à rendre l'entraînement superflu). On observe donc une relative stabilité de $w_1$ dans ce régime ($  \frac{\Vert(w_1^{k+1} - w_1^{k})\Vert}{\Vert w_1^{k}\Vert} \ll 1$). \\
	\end{itemize}
\includegraphics[width=15cm]{"Fig4 Cv w1 2 regimes"}\\


Nous essayons donc dans la suite de nous placer à la frontière entre ces deux régimes.\\

\subsection{Boosting des classifieurs faibles}

Q4. Dans cette partie, nous cherchons une combinaison linéaire des classifieurs faibles entraînés dans la première partie.\\
Pour cela nous itérons à chaque étape sur toutes les images et tous nos classifieurs afin de déterminer le classifieur avec la plus faible erreur pondérée. Nous avons, à ce titre, utilisé $MPI\_MINLOC$ pour déterminer l’indice du meilleur classifieur. A chaque étape, nous avons ensuite mis à jour les poids des images pour rendre moins efficace le classifieur faible sélectionné au cours de cette étape (on a baissé le poids des images qu'il classait déjà correctement) afin de sélectionner d’autres classifieurs par la suite.\\
L’idée est de finalement obtenir une combinaison de classifieurs qui sont efficaces sur l’ensemble des images.\\

Pour mettre en oeuvre cette méthode, nous avons utilisé la fonction $\frac{1}{2}\log(\frac{1-x}{x})$ qui nous permet d'obtenir une nouvelle pondération en fonction de l’erreur pondérée du meilleur classifieur.\\

\begin{center}
\includegraphics[width=10cm]{"Fig5 alpha"}\\
\end{center}

Nous voyons que cette fonction change de signe autour d’une erreur de 0.5 et cela a conduit à la limitation du nombre de classifieurs faibles sélectionnés par notre algorithme. En effet, un classifieur avec une erreur un peu plus faible que cette valeur était sélectionné puis, à cause de la modification des poids, il devenait moins efficace et passait à une erreur supérieure à 0.5 et donc, lorsqu’il était repris, on améliorait cette fois ses performances, le faisant à nouveau passer sous la barre de 0.5 d’erreur pondérée. Ce sueil de l'erreur de 0.5 nous a empêché d’obtenir une combinaison riche de classifieurs puisqu’en général nous avons obtenu moins d’une dizaine de classifieurs, indépendamment de la valeur de N.\\

\section{Test du classifieur final}
\subsection{Analyse des caractéristiques sélectionnées}
Notre programme nous permet d'afficher les informations sur les caractéristiques sélectionnées pour le classifieur final. Nous avons pu afficher ces dernières sur un visage pour une commande de $\textit{boostingDetection 2000 0.1 200 20 0 0.001}$. On obtient en sortie les informations relatives aux caractéristiques sélectionnées:
\begin{center}
  \includegraphics[width=10cm]{"Caracteristiques"}
\end{center}
Reportées sur les images, ces caractéristiques nous donnent les carrés suivants:
\begin{center}
  \includegraphics[width=15cm]{"Cars visage"}
\end{center}
On peut voir que certains classifieurs sont situées sur des zones qui auraient semblé pertinentes a priori (cependant, ces zones ne sont pas toujours exactement au même endroit sur toutes les images) à savoir l'oeil, les coins des yeux et le sourcil.\\

\subsection{Choix de $\theta$}
Q5. Notre classifieur final était très peu sensible à $\theta$. Les paramètres $K=2000$, $\epsilon = 0.1$, $n = 200$, $N = 20$, $\theta = 0$ et une normalisation de $0.001$ nous ont néanmoins permis d’obtenir un $F-score$ de $0.444$ sur toutes les images du dossier test.\\

Voici la courbe du $F-score$ en fonction de $\theta$:
\begin{center}
  \includegraphics[width=10cm]{"Fscore de theta"}
\end{center}
On peut voir qu'une valeur de $\theta$ aux alentours de $-0.4$ semble optimal. On a dans ce cas, 308 vrai positifs contre 252 faux positifs et 4163 vrais négatifs contre 510 faux négatifs.\\
Le graphe des faux négatifs en fonction des faux positifs pour $\theta$ variant enrte -1 et 1 ne nous apprend pas grand chose :
\begin{center}
  \includegraphics[width=10cm]{"fn de fp"}
\end{center}
Mais la courbe ROC (TPR en fonction de FPR) est plus intéressante :
\begin{center}
  \includegraphics[width=10cm]{"ROC"}
\end{center}
On doit aller en dessous de -1 pour obtenir l'ensemble de la courbe (-50) et un peu au dela de 1 (4). On voit que notre classifieur décroche au point $(0.07, 0.4)$ et que l'on devrait essayer de se placer ici pour maximiser son efficacité (très peu de fausses alarmes même s'il n'est pas encore très efficace pour détecter les visages ($TPR = 0.4$) ce qui correspond à des valeurs de $\theta$ entre -1 et 0 comme on peut le voir sur la partie de la courbe ROC correspondant à ces valeurs:
\begin{center}
  \includegraphics[width=10cm]{"ROC zoom"}
\end{center}
\section{Ouverture}
Nous avons cependant été limités dans l’apprentissage par le nombre d’images disponibles (5233 images disponibles dont 84.37$\%$ de négatives), et notamment le fait qu’elles soient identiques dans les 3 dossiers (app, test et dev). Nous aurions peut-être dû passer à des caractéristiques rectangulaires pour obtenir davantage d’informations. Enfin, nous pourrions envisager de diviser la valeur de chaque caractéristique par sa taille afin de normaliser toutes les valeurs et de pouvoir ainsi les comparer entre elles et qu’elles soient toutes de l’ordre de l’unité (pour que $w_1$ et $w_2$ puissent avoir des influences du même ordre de grandeur (cf. 3.3 Choix de $\epsilon$).


\end{document}
